\documentclass{beamer}

\usepackage[pantone3282,english]{wwustyle}

\usepackage[ngerman]{babel}
\usepackage[utf8]{inputenc}
\usepackage[T1]{fontenc}
\usepackage{listings}

% Uncomment the following two lines if you want to prepare your document for
% the fast mode.
% \usetikzlibrary{external}
% \tikzexternalize

\author{Christof, Joschka, Marius, Thomas}
\title{Multiple RTUs}
%\institutelogo{Logo on title frame}
%\institutelogosmall{Logo on other frames}
\subtitle{Sprint 3}

\begin{document}

%%%%%%%%%%%%%%% WWUstyle "fast" mode %%%%%%%%%%%%%%%%%%%%%
% Do the following steps in order to speed up the compilation time of your
% presentation:
%
% 1. Include the externalization tikz library in the preamble of your document.
%    This is always recommended if you are using tikz in your document.
% 2. Uncomment the \wwupreparefastmode command below
% 3. Compile your document with command line option '-shell-escape',
%    e.g.: 'pdflatex -shell-escape beamer.tex'
% 4. Comment (or delete) the \wwupreparefastmode
% 5. Add option 'fast' to the 'wwustyle' package declaration line.
% 6. Be happy!

% \wwupreparefastmode


\begin{frame}[plain]
	\maketitle
\end{frame}

\begin{frame}
	\frametitle{Inhaltsverzeichnis}
	\begin{itemize}
		\item mehrere RTUs
		\item Logik
		\item RTU-Server
		\item WebVis
	\end{itemize}
\end{frame}

\begin{frame}
	\frametitle{mehrere RTUs}
	\begin{alertblock}{Hardcoding ist böse!}
		def \_\_call\_\_(self, **model\_params): \\
		\quad """Call :meth:`create()` to instantiate one model.""" \\
		\quad self.\_check\_params(**model\_params) \\
		\quad return self.create(1, **model\_params)[0] 
	\end{alertblock}
	\pause
	\begin{itemize}
		\item MonitoringRTU als Wrapper-Klasse
		\item Erstellung von RTUs für jede .xml-Datei
		\item jede RTU als eigener Thread
	\end{itemize}
\end{frame}

\begin{frame}
	\frametitle{Logik}
	\begin{itemize}
		\item benutzt korrekte physikalische Werte
		\item R1, R2, R4 und P1 implementiert
		\item Überprüft Kommandos
	\end{itemize}
\end{frame}

\begin{frame}
	\frametitle{RTU-Server}
	\begin{itemize}
		\item bis jetzt:
		  \begin{itemize}
		  	\item ModbusTcp Server
		  	\item Hacker Tools verbinden sich über ModbusTcpClient
		  	\item[$\Rightarrow$] Client kann keine Daten lesen, bevor sie nicht geschrieben wurden!
		  \end{itemize}
		\pause
		\item Plan:
			\begin{itemize}
				\item Verbindung über normalen TcpSocket
				\item Daten werden als Tupel (index, value) gesendet
				\item Server interpretiert die Nachricht und schreibt den Wert an die passende Stelle in den Datablock
			\end{itemize}
	\end{itemize}
\end{frame}

\begin{frame}
	\frametitle{WebVis}
	\begin{itemize}
		\item WebVis wurde als .exe ausgeführt
		\item Reverse Engineering, um Aufbau zu verstehen
		\item gegebene Python-Scripts modifiziert, sodass Visualisierung über diese läuft
		\item[$\Rightarrow$] modifizierbar
		\pause
		\item Hacker Tools verbindet sich mit WebVis-Server
		\item schickt Daten als Liste, welche weitergegeben werden an das Javascript-Script
		\item mosaik.js ändert CSS-Klasse der Nodes/Branches
	\end{itemize}
\end{frame}

\begin{frame}
	\centering
	Danke für Ihre Aufmerksamkeit! \\
	Noch Fragen?
\end{frame}

\end{document}
