\documentclass{beamer}

\usepackage[pantone3282,english]{wwustyle}

\usepackage[ngerman]{babel}
\usepackage[utf8]{inputenc}
\usepackage[T1]{fontenc}
\usepackage{listings}

% Uncomment the following two lines if you want to prepare your document for
% the fast mode.
% \usetikzlibrary{external}
% \tikzexternalize

\author{Christof, Marius, Thomas}
\title{Multiple RTUs}
%\institutelogo{Logo on title frame}
%\institutelogosmall{Logo on other frames}
\subtitle{Sprint 4}

\begin{document}

%%%%%%%%%%%%%%% WWUstyle "fast" mode %%%%%%%%%%%%%%%%%%%%%
% Do the following steps in order to speed up the compilation time of your
% presentation:
%
% 1. Include the externalization tikz library in the preamble of your document.
%    This is always recommended if you are using tikz in your document.
% 2. Uncomment the \wwupreparefastmode command below
% 3. Compile your document with command line option '-shell-escape',
%    e.g.: 'pdflatex -shell-escape beamer.tex'
% 4. Comment (or delete) the \wwupreparefastmode
% 5. Add option 'fast' to the 'wwustyle' package declaration line.
% 6. Be happy!

% \wwupreparefastmode


\begin{frame}[plain]
	\maketitle
\end{frame}

\begin{frame}
	\frametitle{Inhaltsverzeichnis}
	\begin{itemize}
		\item Datenaustausch PyPower-RTU
		\item RTU-Kommunikation
		\item Sensorvalidierung
		\item Hacker-Tools Datenmanipulation
		\item Hacker-Tools Script Interpreter
		\item Ziele für Sprint 5
	\end{itemize}
\end{frame}

\begin{frame}
	\frametitle{Datenaustausch PyPower-RTU}
	\begin{itemize}
		\item für P2, P4
		\item Attribute hinzugefügt:
			\begin{itemize}
				\item demo\_vuln.py: connect()
				\item RTU: Metadaten
			\end{itemize}
		\pause
		\item Problem mit P2, P4
	\end{itemize}
\end{frame}

\begin{frame}
	\frametitle{RTU-Kommunikation}
	\begin{itemize}
		\item jeweils in einem Thread:
			\begin{itemize}
				\item Server-Objekt
				\item Client-Objekt
			\end{itemize}
		\item send-Funktion
			\begin{itemize}
				\item zu speziellen RTUs
				\item zu allen RTUs
				\item[$\Rightarrow$] gibt dict mit Antworten zurück
			\end{itemize}
		\item broadcast-Funktion
			\begin{itemize}
				\item sendet an alle RTUs
				\item nur für Benachrichtigungen
			\end{itemize}
	\end{itemize}
\end{frame}

\begin{frame}
	\frametitle{Sensorvalidierung}
	\begin{itemize}
		\item Inspiration: BLITHE-Paper \\
			(Behaviour Rule-Based Insider Threat Detection for Smart Grids)
		\item trusted-Label
		\item warning-value als Vertrauensstatus eines Sensors
		\item Überprüfung der Veränderung von physikalischen Daten
		\item Überprüfung des Voltage-Angles zwischen benachbarten RTUs
		\item Überprüfung, ob alle Sensoren an einem Node den gleichen Wert haben
	\end{itemize}
\end{frame}

\begin{frame}
	\frametitle{Hacker-Tools Datenmanipulation}
	\begin{itemize}
		\item Hacker-Daten überschreiben Daten der Simulation
		\item Problem:
			\begin{itemize}
				\item Wann kann einem Sensor wieder vertraut werden?
				\item D.h.: Wann sollen wieder die Simulationsdaten verwendet werden?
			\end{itemize}
	\end{itemize}
\end{frame}

\begin{frame}
	\frametitle{Hacker-Tools Script Interpreter}
	\begin{itemize}
		\item für automatische Angriffe
		\item unterstützt Hacker-Tools-Funktionen:
			\begin{itemize}
				\item listservers
				\item connect
				\item listbranches
				\item getstate
				\item setswitch
				\item setsensor
				\item setmaxcurrent
			\end{itemize}
	\end{itemize}
\end{frame}

\begin{frame}
	\frametitle{Hacker-Tools Script Interpreter}
	\begin{itemize}
		\item unterstützt folgende Funktionen:
			\begin{itemize}
				\item set (Variablen)
				\item get
				\item random (range und Arrays)
				\item if - else
				\item for-loop
				\item Logik-Funktionen
					\begin{itemize}
						\item and
						\item or
					\end{itemize}
				\item Vergleichs-Operatoren
					\begin{itemize}
					 	\item ==, <, <=, >, >=, !=
					\end{itemize}
			\end{itemize}
		\item automatische Klammerung von arithmetischen Ausdrücken
	\end{itemize}
\end{frame}

\begin{frame}
	\frametitle{Ziele für Sprint 5}
	\begin{itemize}
		\item Broadcast der RTU, wenn Angriff entdeckt
		\item erweiterter Umgang mit untrustable Sensoren
		\item Angriffsszenarien
	\end{itemize}
\end{frame}

\begin{frame}
	\centering
	Danke für Ihre Aufmerksamkeit! \\
	Noch Fragen?
\end{frame}

\end{document}
